% Options for packages loaded elsewhere
\PassOptionsToPackage{unicode}{hyperref}
\PassOptionsToPackage{hyphens}{url}
%
\documentclass[
]{book}
\usepackage{lmodern}
\usepackage{amssymb,amsmath}
\usepackage{ifxetex,ifluatex}
\ifnum 0\ifxetex 1\fi\ifluatex 1\fi=0 % if pdftex
  \usepackage[T1]{fontenc}
  \usepackage[utf8]{inputenc}
  \usepackage{textcomp} % provide euro and other symbols
\else % if luatex or xetex
  \usepackage{unicode-math}
  \defaultfontfeatures{Scale=MatchLowercase}
  \defaultfontfeatures[\rmfamily]{Ligatures=TeX,Scale=1}
\fi
% Use upquote if available, for straight quotes in verbatim environments
\IfFileExists{upquote.sty}{\usepackage{upquote}}{}
\IfFileExists{microtype.sty}{% use microtype if available
  \usepackage[]{microtype}
  \UseMicrotypeSet[protrusion]{basicmath} % disable protrusion for tt fonts
}{}
\makeatletter
\@ifundefined{KOMAClassName}{% if non-KOMA class
  \IfFileExists{parskip.sty}{%
    \usepackage{parskip}
  }{% else
    \setlength{\parindent}{0pt}
    \setlength{\parskip}{6pt plus 2pt minus 1pt}}
}{% if KOMA class
  \KOMAoptions{parskip=half}}
\makeatother
\usepackage{xcolor}
\IfFileExists{xurl.sty}{\usepackage{xurl}}{} % add URL line breaks if available
\IfFileExists{bookmark.sty}{\usepackage{bookmark}}{\usepackage{hyperref}}
\hypersetup{
  pdftitle={The Knowledge},
  pdfauthor={Geographic Data Science Lab},
  hidelinks,
  pdfcreator={LaTeX via pandoc}}
\urlstyle{same} % disable monospaced font for URLs
\usepackage{longtable,booktabs}
% Correct order of tables after \paragraph or \subparagraph
\usepackage{etoolbox}
\makeatletter
\patchcmd\longtable{\par}{\if@noskipsec\mbox{}\fi\par}{}{}
\makeatother
% Allow footnotes in longtable head/foot
\IfFileExists{footnotehyper.sty}{\usepackage{footnotehyper}}{\usepackage{footnote}}
\makesavenoteenv{longtable}
\usepackage{graphicx}
\makeatletter
\def\maxwidth{\ifdim\Gin@nat@width>\linewidth\linewidth\else\Gin@nat@width\fi}
\def\maxheight{\ifdim\Gin@nat@height>\textheight\textheight\else\Gin@nat@height\fi}
\makeatother
% Scale images if necessary, so that they will not overflow the page
% margins by default, and it is still possible to overwrite the defaults
% using explicit options in \includegraphics[width, height, ...]{}
\setkeys{Gin}{width=\maxwidth,height=\maxheight,keepaspectratio}
% Set default figure placement to htbp
\makeatletter
\def\fps@figure{htbp}
\makeatother
\setlength{\emergencystretch}{3em} % prevent overfull lines
\providecommand{\tightlist}{%
  \setlength{\itemsep}{0pt}\setlength{\parskip}{0pt}}
\setcounter{secnumdepth}{5}
\usepackage{booktabs}
\usepackage{amsthm}
\makeatletter
\def\thm@space@setup{%
  \thm@preskip=8pt plus 2pt minus 4pt
  \thm@postskip=\thm@preskip
}
\makeatother
\usepackage[]{natbib}
\bibliographystyle{apalike}

\title{The Knowledge}
\author{Geographic Data Science Lab}
\date{2020-05-31}

\begin{document}
\maketitle

{
\setcounter{tocdepth}{1}
\tableofcontents
}
\hypertarget{the-knowledge}{%
\chapter{The Knowledge}\label{the-knowledge}}

In preparation to be able to stay productive while having to work fully remote, this document presents a few things jotted down in one place to help with setups, etc.

The first step is a checklist everyone should go through:

\begin{enumerate}
\def\labelenumi{\arabic{enumi}.}
\tightlist
\item
  Do you have a computer to work on at home?
\item
  Are \emph{all} your relevant files accessible from home?
\item
  Do you have a webcam, mic and speakers/headphones?
\item
  Are you able to connect to computers on campus? This involves being setup with the University VPN?
\end{enumerate}

These are the basic elements you will require, so if the answer to any of the above is no, please speak with your PhD supervisor.

\hypertarget{remote-working}{%
\chapter{Remote Working}\label{remote-working}}

\hypertarget{what-is-remote-working}{%
\section{What is Remote Working?}\label{what-is-remote-working}}

``Remote work refers to organizational work that is performed outside of the normal organizational confines of space and time. The term telecommuting refers to the substitution of communications capabilities for travel to a central work location. Office automation technology permits many office workers to be potential telecommuters in that their work can be performed remotely with computer and communications support'' \citep{olson1983remote}

Some of the first trials of remote working date back to the eighties as reported by \citet{olson1983remote}. Since then, the spread and development of Information and Communication Technologies have brought about a significant increase in the popularity of remote work. Nowadays, it is possible to find fully remote jobs advertised particularly in IT and data science.

The current pandemic has forced millions of workers at home, making remote work a necessity rather than an option. An early study on COVID-19 and remote work reported that in the US the fraction of workers who switched to working from home is about 34.1\%, while 14.6\% were already working from home pre-COVID-19 \citep{brynjolfsson2020covid}.

Not all types of work are suitable to be performed from home \citep{holgersen2020and},
but one of the impacts of these dramatic circumstances can be a further increase of the remote work practice.

Computationally intensive tasks can be easily approached in a remote setting by accessing computing resources through the network. This document will guide you in accessing servers located in the Geographic Data Science Lab to perform heavy computations. However, an other increasingly popular option, that many companies are now implementing, is the use of cloud computing infrastructures such as Amazon Web Services (AWS), Salesforce's CRM system, Microsoft Azure.

\hypertarget{advantages-and-disadvantages-of-remote-work}{%
\section{Advantages and Disadvantages of Remote Work}\label{advantages-and-disadvantages-of-remote-work}}

One of the most direct consequences of remote work is changes in commuting behaviours, bringing about time saving and a potential reduction in traffic congestion and air pollution. These were among the main points stressed by the early advocates for remote working, but researches in transportation studies have shown conflicting results. Although reductions in number and length of commuting trips is reported in some of the earlies studies \citep{Kitamura1991, olson1983remote}, more recent research shows that the expectation that home-based telework reduces travel is not so apparent \citep{e2018does} and time saving seems not to be a major pull factor \citep{bailey2002review}.

The higher flexibility afforded by remote working is mentioned as an advantage, particularly for those who would have not taken part of the workforce without such settings because of caring committments \citep{olson1983remote}.

Higher productivity of remote workers has been reported in some studies. However, it has to be noted that productivity and concentration at home are strongly dependent from environmental conditions \citep{bailey2002review}. Inequality in living conditions is an issue that we are clearly seeing during the coronavirus pandemic.

One of the most cited drawbacks of remote working is professional and social isolation \citep{bailey2002review}, which can be also seen as making more difficoult collaborative work and collective workers actions.

\hypertarget{tips}{%
\section{Tips}\label{tips}}

Here a collection of tips that have been shared on the internet on how to avoid burn out and be effective while working from home:

\hypertarget{vpn}{%
\chapter{VPN}\label{vpn}}

\hypertarget{what-is-a-vpn}{%
\section{\texorpdfstring{\emph{What is a VPN?}}{What is a VPN?}}\label{what-is-a-vpn}}

A VPN (virtual private network) connects a machine that lies outside of the university (ie. outside the firewall) to the internal network. When the VPN is running, your network traffic (e.g.~Internet) is routed through the university in the same way as if the computer was on your work desk. This enables you to:

\begin{itemize}
\tightlist
\item
  Access journal websites like you would inside the university
\item
  Access network drives (e.g.~M Drive etc) - but be careful when transferring big files
\item
  Access servers (e.g.~over the terminal / command line / ftp)
\end{itemize}

\hypertarget{how-can-i-setup-the-vpn-for-liverpool}{%
\section{\texorpdfstring{\emph{How can I setup the VPN for Liverpool?}}{How can I setup the VPN for Liverpool?}}\label{how-can-i-setup-the-vpn-for-liverpool}}

You can find a more comprehensive guide to setup your VPN on the
\href{setup_vpn.md}{\texttt{setup\_vpn.md}} document of this repository.

\hypertarget{set-up-a-vpn-connection}{%
\section{Set up a VPN connection}\label{set-up-a-vpn-connection}}

This document describes how to set up a VPN. The text below has been copied from CSD website.

To access the VPN service:

\textbf{1. Register}

Submit \href{https://liverpool.service-now.com/sp?id=sc_cat_item\&sys_id=bd8d37f1376ba60051a3532e53990e3f}{a request to register for the VPN service via CSD}. You will need to explain why you require VPN access and what you intend to do with it. It may be that other services offer the solution you require instead of the VPN.

\textbf{2. Download the VPN Client}

Once your registration is confirmed you will need to download and install the GlobalProtect VPN Client if you are using Windows 10.

You will need admin rights in order to download the GlobalProtect VPN Client. This is how to access admin rights on your MWS PC. Please access admin rights before attempting to download the VPN Client.

The GlobalProtect VPN Client is also compatible with MacOS. Mobile and tablet devices cannot connect using this client. If you are using a Linux distribution, such as Ubuntu, you may be able to connect to the University network using VPNc.

\textbf{3. Open the client and connect}

Click here for instructions on how to use the GlobalProtect VPN Client to access the University network from your PC or Mac.

\hypertarget{how-to-use-the-globalprotect-vpn-client-to-access-the-university-network}{%
\section{How to use the GlobalProtect VPN Client to access the University network}\label{how-to-use-the-globalprotect-vpn-client-to-access-the-university-network}}

You must make sure you are registered to access the VPN service first. Once you are registered you can download and install the GlobalProtect VPN Client to connect to the University network.

GlobalProtect is compatible with Windows 10 and Macs. It is not possible to use GlobalProtect to connect to the University network on a mobile or tablet device.

To install the Client

Access Admin Rights on you PC.

Visit \url{https://vpn.liv.ac.uk}
Enter your University username and password to login to the VPN portal.

Click the appropriate link to download the required version of the VPN client - Windows 32 bit, Windows 64 bit, or Mac OS. (To check which version you require, see your system properties on your device)

Once the file has downloaded, double-click to run the installation. Follow the steps through the installation wizard, accepting the default options.

Once installed you will see the GlobalProtect ``globe'' icon appear in the system tray (bottom right, near the clock). It is a globe and it will have a red x on it, showing that it is not currently connected.

Double-click on the GlobalProtectglobe icon in the system tray. In the window that opens, enter the following:

Username: enter your University username
Password: enter your University password
Portal: vpn.liv.ac.uk

Click Apply.

The GlobalProtect VPN client will then automatically connect to the University network - the red cross should disappear form the icon in the system tray.

You can close the window: the client will stay connected.
To connect and disconnect the client

Once the GlobalConnect VPN client has been installed, the icon will remain in your system tray

To connect the VPN right click the GlobalProtect icon in the system tray and choose Connect.
When you have finished and want to disconnect the VPN, right click on the icon and choose Disconnect.

How to allow third party applications - like Global Protect - to install on a Mac

\begin{enumerate}
\def\labelenumi{\arabic{enumi}.}
\item
  Open System Preferences and click Security \& Privacy
\item
  Select the General tab
\item
  Click the lock in the lower left-hand corner
\item
  Enter your computer username and password, then select Unlock
\end{enumerate}

\hypertarget{ssh}{%
\chapter{SSH}\label{ssh}}

\hypertarget{what-is-ssh}{%
\section{What is SSH?}\label{what-is-ssh}}

Secure Shell (SSH) is a cryptographic network protocol for accessing a computer over an unsecured network. It gives you secure access to a machine's command-line.
Secure Shell provides strong password authentication and public key authentication, as well as encrypted data communications between two computers connecting over an open network, such as the internet. However, all computers within the University of Liverpool Network are not accessible from the open internet for security reasons.
Therefore, to access a machine at the University you do not only need to be connected to the internet, but also to the Virtual Private Network (VPN) that \emph{virtually brings you to the University of Liverpool Network}.

The use of SSH to connect to a remote host is performed through the following command:

\begin{verbatim}
ssh UserName@hostIPaddress 
\end{verbatim}

\begin{verbatim}
The authenticity of host 'hostIPaddress' cannot be established.
 DSA key fingerprint is 01:23:45:67:89:ab:cd:ef:ff:fe:dc:ba:98:76:54:32:10.
 Are you sure you want to continue connecting (yes/no)?
\end{verbatim}

\hypertarget{unix-commands}{%
\section{Unix Commands}\label{unix-commands}}

Servers often run unix operating systems such as GNU/Linux. Unix commands are essential to perform operations from the terminal.

The following are the most frequently used commands:

\hypertarget{list-files}{%
\subsection{List Files:}\label{list-files}}

\begin{verbatim}
user@host:~$ ls -lh
\end{verbatim}

\hypertarget{make-directory}{%
\subsection{Make Directory:}\label{make-directory}}

\begin{verbatim}
user@host:~$ mkdir Fancy_Project
\end{verbatim}

\hypertarget{move}{%
\subsection{Move:}\label{move}}

\begin{verbatim}
user@host:~/Fancy_Project$ mv example.txt new_name.txt  This is to change a file name
user@host:~$ mv example.txt Fancy_Project/ This is to move a file to a folder
user@host:~$ mv -r Fancy_Project/ New_Folder/ This is to move a folder with all its content to an other folder 
\end{verbatim}

\hypertarget{remove}{%
\subsection{Remove:}\label{remove}}

\begin{verbatim}
user@host:~$ rm example.txt 
user@host:~$ rm -r Fancy_Project/
\end{verbatim}

\hypertarget{change-your-working-directory}{%
\subsection{Change your Working Directory:}\label{change-your-working-directory}}

\begin{verbatim}
user@host:~$ cd Fancy_Project/
\end{verbatim}

\hypertarget{copy}{%
\subsection{Copy:}\label{copy}}

\begin{verbatim}
user@host:~$ cp ./path/filename1.txt ./path/filename2.txt
user@host:~$ cp −r ./folder ./destination/
\end{verbatim}

\hypertarget{files}{%
\chapter{Files}\label{files}}

You need to make sure that:

\begin{enumerate}
\def\labelenumi{\alph{enumi})}
\tightlist
\item
  You have access to all of your files
\item
  Your files are backed up so your setup is not entirely reliant on a single device
\item
  Each device on which your files are copied or from which they are accessed
  is encrypted
\end{enumerate}

\hypertarget{backupsync}{%
\section{Backup/sync}\label{backupsync}}

The simplest and recommended way to do this at Liverpool is to keep all your
files and data on your university account at OneDrive. This is part of the
Office 365 Suite available from the university, you can find more info at:

\begin{quote}
\href{https://www.liverpool.ac.uk/csd/working-from-home/}{\texttt{https://www.liverpool.ac.uk/csd/working-from-home/}}
\end{quote}

There are Windows and Mac clients that work relatively well (equivalent to
Dropbox client).

Once you are set up, copy all your files onto your OneDrive account, which
will create a copy of them in Microsoft's secure cloud. The exception is
where you have data that has requirements to be managed in particular ways - e.g.
only from a single machine etc; not in the cloud.

Please, be sure to speak with your PhD supervisor if you access data that
may pose some challenges when moving from local machines or within the university
network (remember OneDrive is in the Cloud, not the university servers!).

\hypertarget{encryption}{%
\section{Encryption}\label{encryption}}

{[}Add guide to encryption: what it is and how to set up on
Windows/macOS/Linux/iOS/Android{]}

\hypertarget{file-transfer-protocol-ftp}{%
\section{File Transfer Protocol (FTP)}\label{file-transfer-protocol-ftp}}

If you need to move large and/or many files from a local machine to a remote
server (e.g.~from your laptop to a Linux machine at the lab), you will
probably want to use something like FTP.

{[}Add FTP guide here{]}

\hypertarget{data-science-stack}{%
\chapter{Data Science Stack}\label{data-science-stack}}

Once you have access from home to all your files and (remote) university computers, next step is easily being able to bootstrap a full data science stack that allows you to carry out scientific work. There are several ways of achieving this, but our preferred strategy is to rely on container technology, in particular on \href{https://www.docker.com/}{Docker}. This will allow you to rapidly install the platform and set of libraries you are familiar with in a way that can be easily reproduced and redeployed (e.g.~on a remote computer on campus).

Here are a series of pages that will help you get a stack ready to go:

\begin{itemize}
\tightlist
\item
  \href{setup_docker.md}{\texttt{setup\_docker.md}}: instructions to install and get Docker
  up and running on different platforms
\item
  \href{setup_jupyterlab.md}{\texttt{setup\_jupyterlab.md}}: instructions to install and
  run a JupyterLab server both on local (e.g.~laptop) and remote (e.g.~server)
  machines
\item
  \href{setup_rstudio.md}{\texttt{setup\_rstudio.md}}: instructions to install and
  run a JupyterLab server both on local (e.g.~laptop) and remote (e.g.~server)
  machines
\end{itemize}

\hypertarget{docker}{%
\chapter{Docker}\label{docker}}

This document describes how to install and use Docker on different platforms.

\hypertarget{installation}{%
\section{Installation}\label{installation}}

If you are on Mac, Linux or Windows 10 Pro/Student editions, installing Docker
is relatively straightforward:

\begin{itemize}
\tightlist
\item
  \href{https://docs.docker.com/docker-for-mac/}{Mac}
\item
  \href{https://docs.docker.com/install/linux/docker-ce/ubuntu/}{Linux official instructions}
\item
  \href{https://docs.docker.com/docker-for-windows/}{Windows 10 Pro/Student}
\end{itemize}

It is important to note that, on Mac and Windows, Docker runs under a virtual
machine so it will not use up all of the resources of your machine
(conversely, it'll equate to be working on a more limited machine). This can
be changed. But if you need more firepower, the idea is that you develop on
your laptop and scale to a server (e.g.~running out of the lab).

The steps to install Docker include:

\begin{itemize}
\tightlist
\item
  Obtain a copy of Docker and install it:

  \begin{itemize}
  \tightlist
  \item
    \texttt{Windows10\ Pro/Enterprise}: \href{https://docs.docker.com/docker-for-windows/install/}{Install Docker Desktop for Windows}
  \item
    \texttt{macOS}: \href{https://docs.docker.com/docker-for-mac/}{Get started with Docker Desktop for Mac}
  \end{itemize}
\item
  Once Docker is successfully installed, make sure to enable access to your main drive (e.g.~\texttt{C:\textbackslash{}\textbackslash{}}):

  \begin{itemize}
  \tightlist
  \item
    \texttt{Windows10\ Pro/Enterprise}: Open the preferences for Docker and click the
    ``Shared Drives'' tab; click on the drive you want to add and then ``Apply''
  \item
    \texttt{macOS}: this feature is automatically enabled
  \end{itemize}
\end{itemize}

\hypertarget{useful-docker-commands}{%
\section{Useful Docker Commands}\label{useful-docker-commands}}

See what containers are running (this also shows you the ID - this is useful to know then R-studio crashes\ldots)

\begin{verbatim}
docker ps
\end{verbatim}

Stop and remove a particular container - replace ID; with the specific ID from the above (listed under CONTAINER ID) - if everything has crashed

\begin{verbatim}
docker stop ID
docker rm -f ID
\end{verbatim}

\hypertarget{jupyterlab}{%
\chapter{JupyterLab}\label{jupyterlab}}

This document shows how to install and run a JupyterLab server locally and
remotely.

\hypertarget{local-install}{%
\section{Local install}\label{local-install}}

This guide assumes you meet the following requirements in your personal
machine (eg. laptop):

\begin{enumerate}
\def\labelenumi{\arabic{enumi}.}
\tightlist
\item
  You have admin rights over your machine
\item
  You are running either Windows 10 Pro, macOS, or Linux
\end{enumerate}

Assuming Docker is up and running (check \href{setup_docker.md}{\texttt{setup\_docker.md}}
for that), you can install an ``image'', which is the install that will let you
run containers, by typing on a command line (\texttt{Terminal.app} or \texttt{PowerShell}
are both good options):

\begin{verbatim}
docker pull darribas/gds:4.0
\end{verbatim}

This will take a while to download but, once finished, you will be ready
to go.

Once the command above has finished installing your GDS stack, you are ready to go! To get a Jupyter session started, you can follow these steps:

\begin{enumerate}
\def\labelenumi{\arabic{enumi}.}
\item
  Run on the same terminal as above the following command:

\begin{verbatim}
docker run --rm -ti -p 8888:8888 -v ${PWD}:/home/jovyan/work darribas/gds:4.0
\end{verbatim}
\end{enumerate}

The command above spins up a container of the \texttt{gds} image, version 4.0 and
ensures it is connected through two main bridges:

\begin{itemize}
\tightlist
\item
  Mapping your laptop's file system from where you have launched the
  command (\texttt{\$\{PWD\}}) to a folder called \texttt{work} on the home directory of
  the container. When you login to Jupyter (see below), you will see a
  \texttt{work} folder and, if you click into it, you should see the content of
  your laptops folder in there.
\item
  Mapping port \texttt{8888} from the container to your laptop, so you can
  connect to it through a browser.
\end{itemize}

It is important to know this command starts a Jupyter server on your machine and keeps it running, so please do not quit the window until you are
done using Jupyter, otherwise it will crash.

\begin{enumerate}
\def\labelenumi{\arabic{enumi}.}
\setcounter{enumi}{1}
\item
  Open your favorite browser (preferably Firefox or Chrome) and point it to
  \texttt{localhost:8888}
\item
  You will be asked for a password or a token. To find the correct one, check
  the terminal where you started the \texttt{docker\ run\ ...} command in 1) and look
  for the long token in the logs. Your prompt should look something (albeit
  not exactly) like this:

\begin{verbatim}
 docker run --rm -ti -p 8888:8888 -v ${PWD}:/home/jovyan/work darribas/gds:4.0
 Executing the command: jupyter notebook
 [I 11:38:40.234 NotebookApp] Writing notebook server cookie secret to /home/jovyan/.local/share/jupyter/runtime/notebook_cookie_secret
 [I 11:38:41.328 NotebookApp] Loading IPython parallel extension
 [I 11:38:41.612 NotebookApp] JupyterLab extension loaded from /opt/conda/lib/python3.7/site-packages/jupyterlab
 [I 11:38:41.612 NotebookApp] JupyterLab application directory is /opt/conda/share/jupyter/lab
 [I 11:38:43.091 NotebookApp] Serving notebooks from local directory: /home/jovyan
 [I 11:38:43.091 NotebookApp] The Jupyter Notebook is running at:
 [I 11:38:43.091 NotebookApp] http://ee20e7549b49:8888/?token=4dc814ee44c64383d5d32dfd439fe62bbc17d9803d9ae434
 [I 11:38:43.091 NotebookApp]  or http://127.0.0.1:8888/?token=4dc814ee44c64383d5d32dfd439fe62bbc17d9803d9ae434
 [I 11:38:43.091 NotebookApp] Use Control-C to stop this server and shut down all kernels (twice to skip confirmation).
 [C 11:38:43.114 NotebookApp]

     To access the notebook, open this file in a browser:
         file:///home/jovyan/.local/share/jupyter/runtime/nbserver-6-open.html
     Or copy and paste one of these URLs:
         http://ee20e7549b49:8888/?token=4dc814ee44c64383d5d32dfd439fe62bbc17d9803d9ae434
      or http://127.0.0.1:8888/?token=4dc814ee44c64383d5d32dfd439fe62bbc17d9803d9ae434
\end{verbatim}

  The token you want to copy is the long series of letter and numbers right
  after \texttt{?token=}, starting by \texttt{4dc814ee}.
\item
  The token should let you into your Jupyter Lab session. Congratulations!
  You can then access the files in your computer through the \texttt{work} directory
  on the left-side pane.
\end{enumerate}

\hypertarget{remote-install}{%
\section{Remote install}\label{remote-install}}

It is also possible to start a Jupyter server as above but, instead of run it
on your local machine, it can run on a remote machine and you connect to that
through your browser over the internet. The process in this context is a bit
more intricate because you need to ensure that the connection is secure, but
overall it follows a similar pattern. The following steps below assume you can
login to the remote server where you want to run Jupyter through \texttt{ssh} and the
serve already has a Docker image installed, ready to be run.

\begin{itemize}
\item
  Login to the remote machine:

\begin{verbatim}
ssh <username>@<server.ip.address>
\end{verbatim}
\end{itemize}

\begin{enumerate}
\def\labelenumi{\arabic{enumi}.}
\item
  Launch the container:

\begin{verbatim}
docker run --rm -ti -p 8888:8888 -v ${PWD}:/home/jovyan/work darribas/gds:4.0 start.sh
\end{verbatim}

  Note we are appending \texttt{start.sh} so it drops us into
  the command line of the container rather than launching the server directly
\item
  Run \texttt{jupyter\ notebook\ -\/-generate-config}
\item
  Generate SSH keys with: \texttt{openssl\ req\ -x509\ -nodes\ -days\ 365\ -newkey\ rsa:2048\ -keyout\ mykey.key\ -out\ mycert.pem}
\item
  Generate password as in the official \href{http://jupyter-notebook.readthedocs.io/en/stable/public_server.html\#preparing-a-hashed-password}{tutorial}
\item
  Update \texttt{/home/jovyan/.jupyter/jupyter\_notebook\_config.py}
  ```python
  \# Set options for certfile, ip, password, and toggle off
  \# browser auto-opening
  c.NotebookApp.certfile = u'/home/jovyan/mycert.pem'
  c.NotebookApp.keyfile = u'/home/jovyan/mykey.key'
  \# Set ip to `\emph{' to bind on all interfaces (ips) for the public server
  c.NotebookApp.ip = '}'
  c.NotebookApp.password = u'sha1:bcd259ccf\ldots{}'
  c.NotebookApp.open\_browser = False

  \# It is a good idea to set a known, fixed port for server access
  c.NotebookApp.port = 8888
  ```
\item
  Launch secure Lab: \texttt{jupyter\ lab}
\item
  On your own machine (laptop/tablet), log in to \texttt{\textless{}server.ip.address\textgreater{}:8888} with the password you have set
\end{enumerate}

\hypertarget{useful-python-docker-images}{%
\section{Useful Python Docker Images}\label{useful-python-docker-images}}

\begin{itemize}
\tightlist
\item
  \href{https://github.com/darribas/gds_env}{\texttt{gds\_env}}: a containerised platform
  for Geographid Data Science in Jupyter (Python \& R)
\item
  \href{https://github.com/jupyter/docker-stacks}{\texttt{jupyter-stacks}}: official
  Jupyter stacks (the \texttt{gds\_env} is based on these)
\end{itemize}

\hypertarget{rstudio-server}{%
\chapter{RStudio server}\label{rstudio-server}}

This guide will help you set up a RStudio server running with Docker. The
benefits of this approach is that it is more reliable and only involves one
install, as opposed to several independent ones.

\hypertarget{local-install-1}{%
\section{Local install}\label{local-install-1}}

If you are an R user; this is a great image to get you up and running with the tidyverse + various geospatial packages + Rstudio server:

\begin{quote}
\href{https://hub.docker.com/r/rocker/geospatial}{\texttt{https://hub.docker.com/r/rocker/geospatial}}
\end{quote}

To install\ldots{}

\begin{verbatim}
docker pull rocker/geospatial
\end{verbatim}

This will require a good internet connection and will take a while, but you
only need to run it once.

Once ready, you can start the instance as follows (Mac example):

\begin{verbatim}
docker run -d --name rstudio -v $HOME:/home/rstudio/alex -e PASSWORD=secret -p 8787:8787 rocker/geospatial
\end{verbatim}

This maps your local home drive (\texttt{\$HOME}) to a given directory in the container (in this case - \texttt{/home/rstudio/alex}).
Using this approach, you can store files on your laptop's drive, and access, edit or create new ones from the container (ie. using RStudio).
The command above also sets up a password (\texttt{secret}) and username (\texttt{rstudio})
to use when you login to RStudio.

On Windows, this is similar (change \texttt{alexa} to your account name - look at your directory structure):

\begin{verbatim}
docker run -d --name rstudio -v  c:\users\alexa:/home/rstudio/alex -e PASSWORD=secret -p 8787:8787 rocker/geospatial
\end{verbatim}

If this runs ok; you access Rstudio server through a browser at: \href{http://localhost:8787}{\texttt{http://localhost:8787}}

\hypertarget{remote-install-1}{%
\section{Remote install}\label{remote-install-1}}

{[}Add here specifics to make this work across the wire in a secure way{]}

\hypertarget{useful-r-docker-images}{%
\section{Useful R Docker Images}\label{useful-r-docker-images}}

  \bibliography{book.bib,packages.bib}

\end{document}
