% Options for packages loaded elsewhere
\PassOptionsToPackage{unicode}{hyperref}
\PassOptionsToPackage{hyphens}{url}
%
\documentclass[
]{book}
\usepackage{lmodern}
\usepackage{amssymb,amsmath}
\usepackage{ifxetex,ifluatex}
\ifnum 0\ifxetex 1\fi\ifluatex 1\fi=0 % if pdftex
  \usepackage[T1]{fontenc}
  \usepackage[utf8]{inputenc}
  \usepackage{textcomp} % provide euro and other symbols
\else % if luatex or xetex
  \usepackage{unicode-math}
  \defaultfontfeatures{Scale=MatchLowercase}
  \defaultfontfeatures[\rmfamily]{Ligatures=TeX,Scale=1}
\fi
% Use upquote if available, for straight quotes in verbatim environments
\IfFileExists{upquote.sty}{\usepackage{upquote}}{}
\IfFileExists{microtype.sty}{% use microtype if available
  \usepackage[]{microtype}
  \UseMicrotypeSet[protrusion]{basicmath} % disable protrusion for tt fonts
}{}
\makeatletter
\@ifundefined{KOMAClassName}{% if non-KOMA class
  \IfFileExists{parskip.sty}{%
    \usepackage{parskip}
  }{% else
    \setlength{\parindent}{0pt}
    \setlength{\parskip}{6pt plus 2pt minus 1pt}}
}{% if KOMA class
  \KOMAoptions{parskip=half}}
\makeatother
\usepackage{xcolor}
\IfFileExists{xurl.sty}{\usepackage{xurl}}{} % add URL line breaks if available
\IfFileExists{bookmark.sty}{\usepackage{bookmark}}{\usepackage{hyperref}}
\hypersetup{
  pdftitle={The Knowledge},
  pdfauthor={Geographic Data Science Lab},
  hidelinks,
  pdfcreator={LaTeX via pandoc}}
\urlstyle{same} % disable monospaced font for URLs
\usepackage{longtable,booktabs}
% Correct order of tables after \paragraph or \subparagraph
\usepackage{etoolbox}
\makeatletter
\patchcmd\longtable{\par}{\if@noskipsec\mbox{}\fi\par}{}{}
\makeatother
% Allow footnotes in longtable head/foot
\IfFileExists{footnotehyper.sty}{\usepackage{footnotehyper}}{\usepackage{footnote}}
\makesavenoteenv{longtable}
\usepackage{graphicx}
\makeatletter
\def\maxwidth{\ifdim\Gin@nat@width>\linewidth\linewidth\else\Gin@nat@width\fi}
\def\maxheight{\ifdim\Gin@nat@height>\textheight\textheight\else\Gin@nat@height\fi}
\makeatother
% Scale images if necessary, so that they will not overflow the page
% margins by default, and it is still possible to overwrite the defaults
% using explicit options in \includegraphics[width, height, ...]{}
\setkeys{Gin}{width=\maxwidth,height=\maxheight,keepaspectratio}
% Set default figure placement to htbp
\makeatletter
\def\fps@figure{htbp}
\makeatother
\setlength{\emergencystretch}{3em} % prevent overfull lines
\providecommand{\tightlist}{%
  \setlength{\itemsep}{0pt}\setlength{\parskip}{0pt}}
\setcounter{secnumdepth}{5}
\usepackage{booktabs}
\usepackage{amsthm}
\makeatletter
\def\thm@space@setup{%
  \thm@preskip=8pt plus 2pt minus 4pt
  \thm@postskip=\thm@preskip
}
\makeatother
\usepackage[]{natbib}
\bibliographystyle{apalike}

\title{The Knowledge}
\author{Geographic Data Science Lab}
\date{2020-06-02}

\begin{document}
\maketitle

{
\setcounter{tocdepth}{1}
\tableofcontents
}
\hypertarget{the-knowledge}{%
\chapter{The Knowledge}\label{the-knowledge}}

In preparation to be able to stay productive while having to work fully remote, this document presents a few things jotted down in one place to help with setups, etc.

The first step is a checklist everyone should go through:

\begin{enumerate}
\def\labelenumi{\arabic{enumi}.}
\tightlist
\item
  Do you have a computer to work on at home?
\item
  Are \emph{all} your relevant files accessible from home?
\item
  Do you have a webcam, mic and speakers/headphones?
\item
  Are you able to connect to computers on campus? This involves being setup with the University VPN?
\end{enumerate}

These are the basic elements you will require, so if the answer to any of the above is no, please speak with your PhD supervisor.

\hypertarget{remote-working}{%
\chapter{Remote Working}\label{remote-working}}

\hypertarget{what-is-remote-working}{%
\section{What is Remote Working?}\label{what-is-remote-working}}

``Remote work refers to organizational work that is performed outside of the normal organizational confines of space and time. The term telecommuting refers to the substitution of communications capabilities for travel to a central work location. Office automation technology permits many office workers to be potential telecommuters in that their work can be performed remotely with computer and communications support'' \citep{olson1983remote}

Some of the first trials of remote working date back to the eighties as reported by \citet{olson1983remote}. Since then, the spread and development of Information and Communication Technologies have brought about a significant increase in the popularity of remote work. Nowadays, it is possible to find fully remote jobs advertised particularly in IT and data science.

The current pandemic has forced millions of workers at home, making remote work a necessity rather than an option. An early study on COVID-19 and remote work reported that in the US the fraction of workers who switched to working from home is about 34.1\%, while 14.6\% were already working from home pre-COVID-19 \citep{brynjolfsson2020covid}.

Not all types of work are suitable to be performed from home \citep{holgersen2020and},
but one of the impacts of these dramatic circumstances can be a further increase of the remote work practice.

Computationally intensive tasks can be easily approached in a remote setting by accessing computing resources through the network. This document will guide you in accessing servers located in the Geographic Data Science Lab to perform heavy computations. However, an other increasingly popular option, that many companies are now implementing, is the use of cloud computing infrastructures such as Amazon Web Services (AWS), Salesforce's CRM system, Microsoft Azure.

\hypertarget{advantages-and-disadvantages-of-remote-work}{%
\section{Advantages and Disadvantages of Remote Work}\label{advantages-and-disadvantages-of-remote-work}}

One of the most direct consequences of remote work is changes in commuting behaviours, bringing about time saving and a potential reduction in traffic congestion and air pollution. These were among the main points stressed by the early advocates for remote working, but researches in transportation studies have shown conflicting results. Although reductions in number and length of commuting trips is reported in some of the earlies studies \citep{Kitamura1991, olson1983remote}, more recent research shows that the expectation that home-based telework reduces travel is not so apparent \citep{e2018does} and time saving seems not to be a major pull factor \citep{bailey2002review}.

The higher flexibility afforded by remote working is mentioned as an advantage, particularly for those who would have not taken part of the workforce without such settings because of caring committments \citep{olson1983remote}.

Higher productivity of remote workers has been reported in some studies. However, it has to be noted that productivity and concentration at home are strongly dependent from environmental conditions \citep{bailey2002review}. Inequality in living conditions is an issue that we are clearly seeing during the coronavirus pandemic.

One of the most cited drawbacks of remote working is professional and social isolation \citep{bailey2002review}, which can be also seen as making more difficoult collaborative work and collective workers actions.

\hypertarget{tips}{%
\section{Tips}\label{tips}}

Here a collection of tips that have been shared on the internet on how to avoid burn out and be effective while working from home:

\hypertarget{vpn}{%
\chapter{VPN}\label{vpn}}

\hypertarget{what-is-a-vpn}{%
\section{\texorpdfstring{\emph{What is a VPN?}}{What is a VPN?}}\label{what-is-a-vpn}}

A VPN (virtual private network) connects a machine that lies outside of the university (ie. outside the firewall) to the internal network. When the VPN is running, your network traffic (e.g.~Internet) is routed through the university in the same way as if the computer was on your work desk. This enables you to:

\begin{itemize}
\tightlist
\item
  Access journal websites like you would inside the university
\item
  Access network drives (e.g.~M Drive etc) - but be careful when transferring big files
\item
  Access servers (e.g.~over the terminal / command line / ftp)
\end{itemize}

\hypertarget{how-can-i-setup-the-vpn-for-liverpool}{%
\section{\texorpdfstring{\emph{How can I setup the VPN for Liverpool?}}{How can I setup the VPN for Liverpool?}}\label{how-can-i-setup-the-vpn-for-liverpool}}

This document describes how to set up a VPN.

To access the VPN service:

\textbf{1. Register}

First you will need to submit \href{https://liverpool.service-now.com/sp?id=sc_cat_item\&sys_id=bd8d37f1376ba60051a3532e53990e3f}{a request to register for the VPN service via CSD}. CSD will need you to explain why you require VPN access and what you intend to do with it.

\textbf{2. Download the VPN Client}

Once your registration is confirmed you will need to download and install a VPN Client.

\textbf{3. Open the client and connect}

Finally you will need to connect to the VPN using your VPN client. Below we provide instructions on how to accomplish this on Windows, Mac and Ubuntu.

\hypertarget{windows-and-mac}{%
\section{Windows and Mac}\label{windows-and-mac}}

You must make sure you are registered to access the VPN service first.
Once you are registered you can download and install the GlobalProtect VPN Client to connect to the University network.
GlobalProtect is compatible with Windows 10 and Macs.
Please access admin rights before attempting to download the VPN Client on your MWS PC.
It is not possible to use GlobalProtect to connect to the University network on a mobile or tablet device.

To install the Client:

\begin{enumerate}
\def\labelenumi{\arabic{enumi})}
\item
  Access Admin Rights on you PC.
\item
  Visit \url{https://vpn.liv.ac.uk}
\item
  Enter your University username and password to login to the VPN portal.
\item
  Click the appropriate link to download the required version of the VPN client - Windows 32 bit, Windows 64 bit, or Mac OS.
  (To check which version you require, see your system properties on your device)
\item
  Once the file has downloaded, double-click to run the installation.
\item
  Follow the steps through the installation wizard, accepting the default options.
\item
  Once installed you will see the GlobalProtect ``globe'' icon appear in the system tray (bottom right, near the clock).
  It is a globe and it will have a red x on it, showing that it is not currently connected.
\item
  Double-click on the GlobalProtectglobe icon in the system tray. In the window that opens, enter the following:

  \begin{itemize}
  \tightlist
  \item
    Username: enter your University username
  \item
    Password: enter your University password
  \item
    Portal: vpn.liv.ac.uk
  \end{itemize}
\item
  Click Apply.
\end{enumerate}

The GlobalProtect VPN client will then automatically connect to the University network - the red cross should disappear form the icon in the system tray.

You can close the window: the client will stay connected.

\hypertarget{to-connect-and-disconnect-the-client}{%
\subsection{To connect and disconnect the client}\label{to-connect-and-disconnect-the-client}}

Once the GlobalConnect VPN client has been installed, the icon will remain in your system tray.

To connect the VPN right click the GlobalProtect icon in the system tray and choose Connect.
When you have finished and want to disconnect the VPN, right click on the icon and choose Disconnect.

How to allow third party applications - like Global Protect - to install on a Mac

\begin{enumerate}
\def\labelenumi{\arabic{enumi})}
\item
  Open System Preferences and click Security \& Privacy
\item
  Select the General tab
\item
  Click the lock in the lower left-hand corner
\item
  Enter your computer username and password, then select Unlock
\end{enumerate}

\hypertarget{ubuntu}{%
\section{Ubuntu}\label{ubuntu}}

The University of Liverpool provides a \href{https://liverpool.service-now.com/sp/?id=kb_article\&sysparm_article=KB0010404\&sys_kb_id=52bfb8303717e200d6fa4a7a43990ef9}{guide}
for setting up VPN on Linux, tested with Ubuntu 14.04, 16.04, 18.04 and Centos 7. The guide recommends the
installation of the VPN Client VPNc:

\begin{verbatim}
sudo apt-get install network-manager-vpnc
sudo apt-get install network-manager-vpnc-gnome
\end{verbatim}

Alternatively yum can be used to install the software:

\begin{verbatim}
sudo yum install NetworkManager-vpnc
sudo yum install NetworkManager-vpnc-gnome
\end{verbatim}

To recap the remaining steps from the \href{https://liverpool.service-now.com/sp/?id=kb_article\&sysparm_article=KB0010404\&sys_kb_id=52bfb8303717e200d6fa4a7a43990ef9}{UoL guide}:

\begin{enumerate}
\def\labelenumi{\arabic{enumi})}
\tightlist
\item
  First you will need to download the UoL VPN configuration file:
  \href{https://pcwww.liv.ac.uk/csd/vpn/University\%20of\%20Liverpool\%20VPN.PCF}{VPN.PCF}
\item
  In the top right corner of your desktop there is a network icon.
  Click on this icon, then choose:
\end{enumerate}

\begin{quote}
VPN Connections \textgreater{} Configure VPN
\end{quote}

\begin{enumerate}
\def\labelenumi{\arabic{enumi})}
\item
  Click ``Add'' and then ``Select Import'' in order to locate your *.pcf file.
\item
  Next enter your MWS username in the User Name field.
\item
  Click the IPv4 Settings tab.
\item
  If left blank fill in the Group password field using the group password specified within the *.pcf file.
\item
  Add DNS servers:

  \begin{itemize}
  \tightlist
  \item
    138.253.110.103
  \item
    138.253.110.104
  \end{itemize}
\end{enumerate}

and search domains (if a search domain field is available):

\begin{itemize}
\tightlist
\item
  liv.ac.uk
\item
  liverpool.ac.uk
\item
  livad.liv.ac.uk
\end{itemize}

\begin{enumerate}
\def\labelenumi{\arabic{enumi})}
\item
  Click on Routes.
\item
  You will need one route with the following details:

  \begin{itemize}
  \tightlist
  \item
    Address: 138.253.0.0
  \item
    Netmask: 255.255.255.0
  \item
    gateway: 0.0.0.0
  \item
    metric: 0
  \end{itemize}
\item
  Tick the ``Use this connection only for resources on its network'' box.
\item
  Click Save / Apply / Close depending on the distribution.
\end{enumerate}

\hypertarget{to-connect-the-client-and-disconnect-the-client}{%
\subsection{To connect the client and disconnect the client:}\label{to-connect-the-client-and-disconnect-the-client}}

To connect to the network, click on the icon in the top right corner of the desktop and choose

\begin{quote}
VPN Connections \textgreater{} University of Liverpool VPN Service
\end{quote}

You will be promted to authenticate using your MWS password. Upon connecting the network
icon will display a locked symbol. You can now use your computera as if you were physically
connected to the University of Liverpool network.

To disconnect click on the network icon again and choose:

\begin{quote}
VPN Connections \textgreater{} Disconnect VPN
\end{quote}

\hypertarget{ssh}{%
\chapter{SSH}\label{ssh}}

\hypertarget{what-is-ssh}{%
\section{What is SSH?}\label{what-is-ssh}}

Secure Shell (SSH) is a cryptographic network protocol for accessing a computer over an unsecured network. It gives you secure access to a machine's command-line.
Secure Shell provides strong password authentication and public key authentication, as well as encrypted data communications between two computers connecting over an open network, such as the internet. However, all computers within the University of Liverpool Network are not accessible from the open internet for security reasons.
Therefore, to access a machine at the University you do not only need to be connected to the internet, but also to the Virtual Private Network (VPN) that \emph{virtually brings you to the University of Liverpool Network}.

\hypertarget{connecting-to-a-remote-host}{%
\section{Connecting to a remote host}\label{connecting-to-a-remote-host}}

The use of SSH to connect to a remote host is performed through the following command:

\begin{verbatim}
ssh <username>@<server.ip.address>
\end{verbatim}

If you are connecting to a server for the first time, then you may receive the following warning:

\begin{verbatim}
The authenticity of host 'hostIPaddress' cannot be established.
 DSA key fingerprint is 01:23:45:67:89:ab:cd:ef:ff:fe:dc:ba:98:76:54:32:10.
 Are you sure you want to continue connecting (yes/no)?
\end{verbatim}

Windows users need to install an SSH client in order to access a server remotely through SSH. There are several clients available, MobaXterm is the option we advise to employ as it provides a number of useful functions for remote computing in a single application.
The free edition can be downloaded \href{https://mobaxterm.mobatek.net/download.html}{here}. To establish a connection with your remote server you open mobaXterm and click on session and then SSH. The Remote Host is the machine you want to access which can be identified by an IP address. IP addresses of our shared resources are listed in - add file name here-.
\href{https://mobaxterm.mobatek.net/demo.html}{Here} you can find a simple demo of the SSH client.

If this is your first time connecting to the server, or if the server has recently been
reconfigured with a new key, then the above message is perfectly normal. You can proceed by typing yes and enter.

Once you have accessed the server the first thing to do is to change the temprorary password we assigned to your user.

\begin{verbatim}
user@host:~$ passwd
\end{verbatim}

\hypertarget{unix-commands}{%
\section{Unix Commands}\label{unix-commands}}

Servers often run unix operating systems such as GNU/Linux. Unix commands are essential to perform operations from the terminal.

The following are the most frequently used commands:

\hypertarget{list-files}{%
\subsection{List Files:}\label{list-files}}

\begin{verbatim}
user@host:~$ ls -lh
\end{verbatim}

In the above example two flags have been added to the ls command:

\begin{itemize}
\tightlist
\item
  -l List with long format, e.g., show file read/write/execute permissions.
\item
  -h List files with readable file size, e.g.~MB, GB, etc.
\end{itemize}

\hypertarget{make-directory}{%
\subsection{Make Directory:}\label{make-directory}}

A new directory can be created using the mkdir command:

\begin{verbatim}
user@host:~$ mkdir <new_directory_name>
\end{verbatim}

\hypertarget{move}{%
\subsection{Move:}\label{move}}

The mv command command can be used to either rename or move files and folders:

\begin{verbatim}
user@host:~$ mv <current_filename> <new_filename>  # This is to change a file name
user@host:~$ mv <filepath> <target_directory> # This is to move a file to a folder
user@host:~$ mv -r <directory_path> <target_directory> # This is to move a folder with all its content to an other folder 
\end{verbatim}

We add the -r (recursive) flag to move a directory and all its contents (subdirectories and files).

\hypertarget{remove}{%
\subsection{Remove:}\label{remove}}

Files and folders and can be deleted using the rm (remove) command:

\begin{verbatim}
user@host:~$ rm <filepath> # To delete a file
user@host:~$ rm -r <directory> # To delete a folder and its content
\end{verbatim}

Again, we add the -r (recursive) flag to remove a directory and all its contents (subdirectories and files).

\textbf{Warning: rm is to be used with caution. There is no trash folder
from which the files can be recovered. Upon using this command the files are deleted.}

\hypertarget{change-your-working-directory}{%
\subsection{Change your Working Directory:}\label{change-your-working-directory}}

To change your working directory use the cd (change directory) command, specifying your target directory:

\begin{verbatim}
user@host:~$ cd <target_directory>
\end{verbatim}

\hypertarget{copy}{%
\subsection{Copy:}\label{copy}}

To copy files use cp.~Again the -r flag can be added to recursively copy all files and subdirectories within a directory:

\begin{verbatim}
user@host:~$ cp <filename> <filecopy>
user@host:~$ cp −r <directory> <target>
\end{verbatim}

\hypertarget{command-line-editors}{%
\section{Command Line Editors}\label{command-line-editors}}

There exists a large number of \href{https://en.wikipedia.org/wiki/List_of_text_editors}{command line editors}
that can be used to edit files directly within the terminal.
To create a file from the terminal you can simply type the following command:

\begin{verbatim}
user@host:~$ > filename.txt
\end{verbatim}

Below we provide instructions for using the
\href{https://www.nano-editor.org/}{nano editor}. To open a file in nano run the following command inside the
terminal:

\begin{verbatim}
user@host:~$ nano <filename>
\end{verbatim}

Once the file opens you can move your cursor using the arrow keys, and edit content
as in any text editor. You can save any changes that you have made to the file
using \textbf{Ctrl + O}. To close the editor press \textbf{Ctrl + X}. Before the editor
closes you will be asked if you want to save your changes. Type Y for Yes to
save changes, and N for No if you want to close the editor without saving.

For an overview of nano shortcut keys you can press \textbf{Ctrl + G}, which will output the following list.

\hypertarget{practical}{%
\section{Practical}\label{practical}}

\begin{enumerate}
\def\labelenumi{\arabic{enumi})}
\item
  Use SSH to access one of the lab servers (the IP address of the server you have been assigned will be provided).
\item
  Change the current temporary password with a password of your choice.
\item
  Create a directory with your project name.
\item
  List files to check that the directory is created.
\item
  Change your working directory with the project directory.
\item
  Create a file, edit and save using nano.
\end{enumerate}

\hypertarget{files}{%
\chapter{Files}\label{files}}

You need to make sure that:

\begin{enumerate}
\def\labelenumi{\alph{enumi})}
\tightlist
\item
  You have access to all of your files;
\item
  Your files are backed up so your setup is not entirely reliant on a single device;
\item
  Each device on which your files are copied or from which they are accessed
  is encrypted.
\end{enumerate}

\hypertarget{backupsync}{%
\section{Backup/sync}\label{backupsync}}

The simplest and recommended way to do this at Liverpool is to keep all your
files and data on your university account at OneDrive. This is part of the
Office 365 Suite available from the university, you can find more info at:

\begin{quote}
\href{https://www.liverpool.ac.uk/csd/working-from-home/}{\texttt{https://www.liverpool.ac.uk/csd/working-from-home/}}
\end{quote}

There are Windows and Mac clients that work relatively well (equivalent to
Dropbox client).

Once you are set up, copy all your files onto your OneDrive account, which
will create a copy of them in Microsoft's secure cloud. The exception is
where you have data that has requirements to be managed in particular ways - e.g.
only from a single machine etc; not in the cloud.

Please, be sure to speak with your PhD supervisor if you access data that
may pose some challenges when moving from local machines or within the university
network (remember OneDrive is in the Cloud, not the university servers!).

\hypertarget{encryption}{%
\section{Encryption}\label{encryption}}

Disk encryption helps protect data on your devices through converting
them into an unreadable format. Deciphering the data without access to the
required keys is challenging. Therefore, should your devices be lost or stolen,
encrypting your devices therefore introduces an additional barrier for someone
attempting to access potentially sensitive data. Please note that, as per
University of Liverpool guidelines,
\href{https://www.liverpool.ac.uk/csd/security/information-security/encryption/}{``the security of confidential information is the responsibility of the individual
member of staff or student NOT the University, nor the line manager or Head of
Department''}.
Encryption methods are platform dependent. A list of relevant guides is provided below:

\begin{enumerate}
\def\labelenumi{\arabic{enumi})}
\tightlist
\item
  \href{https://www.liverpool.ac.uk/csd/security/information-security/encryption/}{Windows}
\item
  \href{https://support.apple.com/en-gb/guide/disk-utility/dskutl35612/mac}{Mac}
\item
  \href{https://help.ubuntu.com/community/Full_Disk_Encryption_Howto_2019}{Ubuntu}
\item
  \href{https://support.apple.com/en-gb/guide/security/secf6276da8a/1/web/1}{iOS}
\item
  \href{https://www.howtogeek.com/141953/how-to-encrypt-your-android-phone-and-why-you-might-want-to/}{Android}
\end{enumerate}

\hypertarget{file-transfer}{%
\section{File Transfer}\label{file-transfer}}

Below we shall discuss two approaches that can be used to transfer files between
two servers:

\begin{enumerate}
\def\labelenumi{\arabic{enumi})}
\tightlist
\item
  \href{https://en.wikipedia.org/wiki/File_Transfer_Protocol}{File Transfer Protocol (FTP)};
\item
  \href{https://en.wikipedia.org/wiki/Secure_copy}{Secure Copy Protocol (scp)}.
\end{enumerate}

\hypertarget{file-transfer-protocol-ftp}{%
\subsection{File Transfer Protocol (FTP)}\label{file-transfer-protocol-ftp}}

If you need to move large and/or many files from a local machine to a remote
server (e.g.~from your laptop to a Linux machine at the lab), you can do so
using a drag and drop interface with an FTP client (e.g., \href{https://filezilla-project.org/}{filezilla}
for Windows/Mac/Ubuntu or \href{https://winscp.net/eng/index.php}{WinSCP} for Windows). To access
a remote server you will need to enter the following into respective fields within your FTP client:

\begin{itemize}
\tightlist
\item
  Host (Remote Server IP Address);
\item
  Username (Your username on the remote host);
\item
  Password (Remote host password).
\end{itemize}

\hypertarget{secure-copy-protocol-scp}{%
\subsection{Secure Copy Protocol (scp)}\label{secure-copy-protocol-scp}}

Alternatively Mac and Linux users can copy files between servers using the scp command.
To copy a local file to a remote server:

\begin{verbatim}
scp <filepath> <username>@<server.ip.address>:<target_directory>
\end{verbatim}

This command can also be use to copy a file from the remote server to your local machine:

\begin{verbatim}
scp <username>@<server.ip.address>:<filepath> <local_target_directory>/
\end{verbatim}

As with copying file locally -r can be added to the above command to recursively copy
all files within a directory. However, if a directory contains a large number of files
then zip your directory before executing scp. The zip file can be unzipped using:

\begin{verbatim}
unzip <filename>.zip -d <target_directory>
\end{verbatim}

Tar files meanwhile can be extracted using:

\begin{verbatim}
tar -xvzf <filename>.tar.gz -C <target_directory>
\end{verbatim}

\hypertarget{file-download}{%
\section{File Download}\label{file-download}}

Often large datasets, etc can be downloaded from the web directly.
The wget command can be used to download files from both http(s)

\begin{verbatim}
wget '<file_url>'
\end{verbatim}

and ftp servers:

\begin{verbatim}
wget -r 'ftp://<username>:<password>@<server.ip.address>/<directory>'
\end{verbatim}

Once the data has been downloaded we must verify that the integrity of the file.
Typically the websites from which data can be downloaded provide a md5 checksum.
This allows us to verify that a file has not been changed:

\begin{verbatim}
md5sum <filename>
\end{verbatim}

\hypertarget{practical-1}{%
\section{Practical}\label{practical-1}}

\hypertarget{file-transfer-1}{%
\subsection{File Transfer}\label{file-transfer-1}}

\begin{enumerate}
\def\labelenumi{\arabic{enumi})}
\tightlist
\item
  Create a file named ``Test.txt'' locally and enter some random text.
  From your local machine copy ``Test.txt'' to your remote server using either
  scp or a ftp client (e.g., using \href{https://winscp.net/eng/index.php}{WinSCP}
  or \href{https://filezilla-project.org/}{filezilla}).
\item
  Upon transferring the file, ssh into the remote server and verify that the file is within the
  specified target directory using the ls command.
\item
  (Optional) Open the file using a
  \href{https://en.wikipedia.org/wiki/List_of_text_editors}{command line editing interface} from the terminal, e.g, nano or vim:
\end{enumerate}

\begin{verbatim}
nano ./Test.txt
\end{verbatim}

\hypertarget{file-download-1}{%
\subsection{File Download}\label{file-download-1}}

For the second task we shall download the CIFAR-10 dataset from
\url{https://www.cs.toronto.edu/~kriz/cifar.html}
using the wget command.

\begin{enumerate}
\def\labelenumi{\arabic{enumi})}
\tightlist
\item
  First we will need to determine the url from which we can download the dataset.
  Visit the \href{https://www.cs.toronto.edu/~kriz/cifar.html}{CIFAR-10 website}, right
  click on the ``CIFAR-10 python version'', and choose the ``copy link'' option,
  which will copy the link to your clipboard.
\item
  Next, ssh into your remote server.
\item
  Type ``wget'' and paste the url into to the terminal by pressing (ctrl + shift + V).
\item
  Press enter to start the file download.
\item
  Upon downloading the CIFAR-10 dataset verify that the md5 checksum matches the one specified on the website.
\end{enumerate}

\hypertarget{data-science-stack}{%
\chapter{Data Science Stack}\label{data-science-stack}}

Once you have access from home to all your files and (remote) university computers, next step is easily being able to bootstrap a full data science stack that allows you to carry out scientific work. There are several ways of achieving this, but our preferred strategy is to rely on container technology, in particular on \href{https://www.docker.com/}{Docker}. This will allow you to rapidly install the platform and set of libraries you are familiar with in a way that can be easily reproduced and redeployed (e.g.~on a remote computer on campus).

Here are a series of pages that will help you get a stack ready to go:

\begin{itemize}
\tightlist
\item
  \href{06_docker.md}{\texttt{setup\_docker.md}}: instructions to install and get Docker
  up and running on different platforms
\item
  \href{07_jupyterlab.md}{\texttt{setup\_jupyterlab.md}}: instructions to run a JupyterLab server within a Docker container both on local (e.g.~laptop) and remote (e.g.~server) machines
\item
  \href{08_RStudio.md}{\texttt{setup\_rstudio.md}}: instructions to run Rstudio server within a Docker container both on local (e.g.~laptop) and remote (e.g.~server) machines.
\end{itemize}

\hypertarget{docker}{%
\chapter{Docker}\label{docker}}

This document describes how to install and use Docker on different platforms.

\hypertarget{installation}{%
\section{Installation}\label{installation}}

If you are on Mac, Linux or Windows 10 Pro/Student editions, installing Docker
is relatively straightforward:

\begin{itemize}
\tightlist
\item
  \href{https://docs.docker.com/docker-for-mac/}{Mac}
\item
  \href{https://docs.docker.com/install/linux/docker-ce/ubuntu/}{Linux official instructions}
\item
  \href{https://docs.docker.com/docker-for-windows/}{Windows 10 Pro/Student}
\end{itemize}

It is important to note that, on Mac and Windows, Docker runs under a virtual
machine so it will not use up all of the resources of your machine
(conversely, it'll equate to be working on a more limited machine). This can
be changed. But if you need more firepower, the idea is that you develop on
your laptop and scale to a server (e.g.~running out of the lab).

The steps to install Docker include:

\begin{itemize}
\tightlist
\item
  Obtain a copy of Docker and install it:

  \begin{itemize}
  \tightlist
  \item
    \texttt{Windows10\ Pro/Enterprise}: \href{https://docs.docker.com/docker-for-windows/install/}{Install Docker Desktop for Windows}
  \item
    \texttt{macOS}: \href{https://docs.docker.com/docker-for-mac/}{Get started with Docker Desktop for Mac}
  \end{itemize}
\item
  Once Docker is successfully installed, make sure to enable access to your main drive (e.g.~\texttt{C:\textbackslash{}\textbackslash{}}):

  \begin{itemize}
  \tightlist
  \item
    \texttt{Windows10\ Pro/Enterprise}: Open the preferences for Docker and click the
    ``Shared Drives'' tab; click on the drive you want to add and then ``Apply''
  \item
    \texttt{macOS}: this feature is automatically enabled
  \end{itemize}
\end{itemize}

\hypertarget{image-and-container}{%
\section{Image and Container}\label{image-and-container}}

Docker can be seen as a tool to generate a computer within your computer (the host). This creates a working environment that employs the host's resources but follows its own internal rules.

\textbf{Image} and \textbf{Container} are two key concepts to understand Docker processes.

\textbf{Images} are files containing all the instructions to build a complete and executable version of an application, relying on the host OS kernel.

\textbf{Containers} are instantiations of images, meaning that they are instances of the images running in an isolated environment. The same image can be instantiated in multiple containers.

\hypertarget{useful-docker-commands}{%
\section{Useful Docker Commands}\label{useful-docker-commands}}

List available images :

\begin{verbatim}
docker image ls
\end{verbatim}

See what containers are running (this also shows you the ID and which port is occupying):

\begin{verbatim}
docker ps
\end{verbatim}

The run command instantiates an image in an isolated container. The generic command is:

\begin{verbatim}
docker run [OPTIONS] IMAGE[:TAG] [COMMAND] [ARG...]
\end{verbatim}

Options that are generally added are:

\begin{verbatim}
--detach or -d -> Run container in background and print container ID
\end{verbatim}

If you do not add this option to the run command you can detach from a running docker session without exiting the shell, the escape sequence \emph{CTRL + p} followed by \emph{CTRL + q} can be used. To re-attach the terminal to the docker session enter: docker attach

\begin{verbatim}
-rm  -> Automatically remove the container when it exits
\end{verbatim}

--publish or -p -\textgreater{} Publish a container's port(s) to the host (This is key when running server-based application through Docker. A network port is the communication enpoint in a client-server model, specifying which program a communication is intended for.)

\begin{verbatim}
--volume or -v -> Bind mount a volume 
\end{verbatim}

All other options are listed \href{https://docs.docker.com/engine/reference/commandline/run/}{here}

To stop the container you have been running you need to send the following commands:

if you have not added -d to the run command, detach from the container with the escape sequence.

\begin{verbatim}
docker ps   to get the Container ID
docker stop <container-ID>  
\end{verbatim}

if you have not added -rm option you need to manually remove the container

\begin{verbatim}
docker rm -f <container-ID>
\end{verbatim}

\hypertarget{jupyterlab}{%
\chapter{JupyterLab}\label{jupyterlab}}

This document shows how to install and run a JupyterLab server locally and
remotely.

\hypertarget{run-jupyter-lab-locally}{%
\section{Run Jupyter Lab Locally}\label{run-jupyter-lab-locally}}

This guide assumes you meet the following requirements in your personal
machine (eg. laptop):

\begin{enumerate}
\def\labelenumi{\arabic{enumi}.}
\tightlist
\item
  You have admin rights over your machine
\item
  You are running either Windows 10 Pro, macOS, or Linux
\end{enumerate}

Assuming Docker is up and running (check \href{setup_docker.md}{\texttt{setup\_docker.md}}
for that), you can install an ``image'', which is the install that will let you
run containers, by typing on a command line (\texttt{Terminal.app} or \texttt{PowerShell}
are both good options):

\begin{verbatim}
docker pull darribas/gds:4.0
\end{verbatim}

This will take a while to download but, once finished, you will be ready
to go.

Once the command above has finished installing your GDS stack, you are ready to go! To get a Jupyter session started, you can follow these steps:

\begin{enumerate}
\def\labelenumi{\arabic{enumi}.}
\item
  Run on the same terminal as above the following command:

\begin{verbatim}
docker run --rm -ti --user root -e NB_UID=$UID -e NB_GID=100 -p 8888:8888 -v ${PWD}:/home/jovyan/work darribas/gds:4.0
\end{verbatim}
\end{enumerate}

The command above spins up a container of the \texttt{gds} image, version 4.0 and
ensures it is connected through two main bridges:

\begin{itemize}
\tightlist
\item
  Mapping your laptop's file system from where you have launched the
  command (\texttt{\$\{PWD\}}) to a folder called \texttt{work} on the home directory of
  the container. When you login to Jupyter (see below), you will see a
  \texttt{work} folder and, if you click into it, you should see the content of
  your laptops folder in there.
\item
  Mapping port \texttt{8888} from the container to your laptop, so you can
  connect to it through a browser.
\end{itemize}

It is important to know this command starts a Jupyter server on your machine and keeps it running, so please do not quit the window until you are
done using Jupyter, otherwise it will crash.

\begin{enumerate}
\def\labelenumi{\arabic{enumi}.}
\setcounter{enumi}{1}
\item
  Open your favorite browser (preferably Firefox or Chrome) and point it to
  \texttt{localhost:8888}
\item
  You will be asked for a password or a token. To find the correct one, check
  the terminal where you started the \texttt{docker\ run\ ...} command in 1) and look
  for the long token in the logs. Your prompt should look something (albeit
  not exactly) like this:

\begin{verbatim}
 docker run --rm -ti -p 8888:8888 -v ${PWD}:/home/jovyan/work darribas/gds:4.0
 Executing the command: jupyter notebook
 [I 11:38:40.234 NotebookApp] Writing notebook server cookie secret to /home/jovyan/.local/share/jupyter/runtime/notebook_cookie_secret
 [I 11:38:41.328 NotebookApp] Loading IPython parallel extension
 [I 11:38:41.612 NotebookApp] JupyterLab extension loaded from /opt/conda/lib/python3.7/site-packages/jupyterlab
 [I 11:38:41.612 NotebookApp] JupyterLab application directory is /opt/conda/share/jupyter/lab
 [I 11:38:43.091 NotebookApp] Serving notebooks from local directory: /home/jovyan
 [I 11:38:43.091 NotebookApp] The Jupyter Notebook is running at:
 [I 11:38:43.091 NotebookApp] http://ee20e7549b49:8888/?token=4dc814ee44c64383d5d32dfd439fe62bbc17d9803d9ae434
 [I 11:38:43.091 NotebookApp]  or http://127.0.0.1:8888/?token=4dc814ee44c64383d5d32dfd439fe62bbc17d9803d9ae434
 [I 11:38:43.091 NotebookApp] Use Control-C to stop this server and shut down all kernels (twice to skip confirmation).
 [C 11:38:43.114 NotebookApp]

     To access the notebook, open this file in a browser:
         file:///home/jovyan/.local/share/jupyter/runtime/nbserver-6-open.html
     Or copy and paste one of these URLs:
         http://ee20e7549b49:8888/?token=4dc814ee44c64383d5d32dfd439fe62bbc17d9803d9ae434
      or http://127.0.0.1:8888/?token=4dc814ee44c64383d5d32dfd439fe62bbc17d9803d9ae434
\end{verbatim}

  The token you want to copy is the long series of letter and numbers right
  after \texttt{?token=}, starting by \texttt{4dc814ee}.
\item
  The token should let you into your Jupyter Lab session. Congratulations!
  You can then access the files in your computer through the \texttt{work} directory
  on the left-side pane.
\end{enumerate}

\hypertarget{run-jupyter-lab-remotely}{%
\section{Run Jupyter Lab Remotely}\label{run-jupyter-lab-remotely}}

It is also possible to start a Jupyter server as above but, instead of run it
on your local machine, it can run on a remote machine and you connect to that
through your browser over the internet. The process in this context is a bit
more intricate because you need to ensure that the connection is secure, but
overall it follows a similar pattern. The following steps below assume you can
login to the remote server where you want to run Jupyter through \texttt{ssh} and the
serve already has a Docker image installed, ready to be run.

\begin{itemize}
\item
  Login to the remote machine:

\begin{verbatim}
ssh <username>@<server.ip.address>
\end{verbatim}
\end{itemize}

\begin{enumerate}
\def\labelenumi{\arabic{enumi}.}
\item
  Launch the container:

\begin{verbatim}
docker run --rm -ti --user root -e NB_UID=$UID -e NB_GID=100 -p 8889:8888 -v ${PWD}:/home/jovyan/work darribas/gds:4.1 start.sh  
\end{verbatim}

  Note we are appending \texttt{start.sh} so it drops us into
  the command line of the container rather than launching the server directly
\item
  Run \texttt{jupyter\ notebook\ -\/-generate-config}
\item
  Generate SSH keys with: \texttt{openssl\ req\ -x509\ -nodes\ -days\ 365\ -newkey\ rsa:2048\ -keyout\ mykey.key\ -out\ mycert.pem}
\item
  Generate password as in the official \href{http://jupyter-notebook.readthedocs.io/en/stable/public_server.html\#preparing-a-hashed-password}{tutorial}
\item
  Update \texttt{/home/jovyan/.jupyter/jupyter\_notebook\_config.py}
  ```python
  \# Set options for certfile, ip, password, and toggle off
  \# browser auto-opening
  c.NotebookApp.certfile = u'/home/jovyan/mycert.pem'
  c.NotebookApp.keyfile = u'/home/jovyan/mykey.key'
  \# Set ip to `\emph{' to bind on all interfaces (ips) for the public server
  c.NotebookApp.ip = '}'
  c.NotebookApp.password = u'sha1:bcd259ccf\ldots{}'
  c.NotebookApp.open\_browser = False

  \# It is a good idea to set a known, fixed port for server access
  c.NotebookApp.port = 8888
  ```
\item
  Launch secure Lab: \texttt{jupyter\ lab}
\item
  On your own machine (laptop/tablet), log in to \texttt{\textless{}server.ip.address\textgreater{}:8888} with the password you have set
\end{enumerate}

\hypertarget{self-signed-certificate-warnings}{%
\subsection{Self-signed Certificate Warnings}\label{self-signed-certificate-warnings}}

Upon accessing the notebook server your browser might warn you that your self-signed certificate is
insecure or unrecognized. A fully compliant self-signed certificate is required to prevent these warnins.
One approach towards solving this issue is to acquire a free SSL certificate via
\href{https://jupyter-notebook.readthedocs.io/en/stable/public_server.html\#using-let-s-encrypt}{Let's Encrypt}.

\hypertarget{useful-python-docker-images}{%
\section{Useful Python Docker Images}\label{useful-python-docker-images}}

\begin{itemize}
\tightlist
\item
  \href{https://github.com/darribas/gds_env}{\texttt{gds\_env}}: a containerised platform
  for Geographid Data Science in Jupyter (Python \& R)
\item
  \href{https://github.com/jupyter/docker-stacks}{\texttt{jupyter-stacks}}: official
  Jupyter stacks (the \texttt{gds\_env} is based on these)
\end{itemize}

\hypertarget{practical-2}{%
\section{Practical}\label{practical-2}}

Follow the steps in the Remote Install section to run jupyter lab on one of the remote servers.

Note that you will need to use the command line editing software nano to edit update jupyter\_notebook\_config.py:

\begin{verbatim}
nano /home/jovyan/.jupyter/jupyter_notebook_config.py
\end{verbatim}

A guide to using nano can be found \href{https://www.howtogeek.com/howto/42980/the-beginners-guide-to-nano-the-linux-command-line-text-editor/}{here}.

\hypertarget{docker-containers-for-r}{%
\chapter{Docker Containers for R}\label{docker-containers-for-r}}

A widely-used suite of docker images for R has been developed by the Rocker project \citep{RJ-2017-065}. 29 repositories of rocker's images are listed on the \href{https://hub.docker.com/u/rocker/}{docker hub} providing R environments customized to perform a variety of tasks. Most of them include R studio server, allowing to work remotely from the R studio server web interface accessible through any browser.

Rocker images are versioned, meaning that if you want to employ a specific version of R rather than the latest you can specify an R version tag in the image name, i.e.~\emph{rocker/verse:3.4.0}, if no tag is requested you will automatically pull the latest version \citep{RJ-2017-065}.

\hypertarget{run-docker-locally}{%
\section{Run Docker Locally}\label{run-docker-locally}}

To run one of the rocker images locally on your laptop you need to install docker as described in - par ref - and pull the image of your choise from docker hub.

A great image with the tidyverse + various geospatial packages + Rstudio server is rocker geospatial:

\begin{quote}
\href{https://hub.docker.com/r/rocker/geospatial}{\texttt{https://hub.docker.com/r/rocker/geospatial}}
\end{quote}

You can pull the repository running the following command:

\begin{verbatim}
docker pull rocker/geospatial
\end{verbatim}

This will require a good internet connection and will take a while, but you
only need to run it once.

Once the image is on your local machine, you can run it with the following command:

\begin{verbatim}
docker run -d --name rstudio -v ${PWD}:/data -e PASSWORD=secret -p 8787:8787 rocker/geospatial
\end{verbatim}

Using this approach, you can store files on your laptop's drive, and access, edit or create new ones from the container (ie. using RStudio).
The command above also sets up a password (\texttt{secret}) and username (\texttt{rstudio})
to use when you login to RStudio.

On Windows, this is similar (change \texttt{alexa} to your account name - look at your directory structure):

\begin{verbatim}
docker run -d --name rstudio -v  c:\users\alexa:/home/rstudio/alex -e PASSWORD=secret -p 8787:8787 rocker/geospatial
\end{verbatim}

If this runs ok; you access Rstudio server through a browser at: \href{http://localhost:8787}{\texttt{http://localhost:8787}}

\hypertarget{run-docker-remotely}{%
\section{Run Docker Remotely}\label{run-docker-remotely}}

Docker is installed on all our shared resources. Most machines already have images that can be used for gds. Rocker/geospatial has been extended to have all libraries that GDSL members generally use -link to the image-.

Once you accessed the server you can list the existing images with the following command:

\begin{verbatim}
docker images ls
\end{verbatim}

check ports

run command same specifics of user names and psw.

\hypertarget{best-practices}{%
\chapter{Best Practices}\label{best-practices}}

\hypertarget{booking-a-server}{%
\section{Booking a server}\label{booking-a-server}}

Before booking a server look at the server information sheet to verify that the resource can meet your needs.

Book the Server with our booking system detailing your name, short description of tasks, forseen resources used (i.e.~gpu, cpu threads,ram) - the booking system will be either an excel file or a shared calendar.

\hypertarget{docker-considerations}{%
\section{Docker Considerations}\label{docker-considerations}}

\begin{itemize}
\item
  Always check what are the existing images before pulling a new one with:

\begin{verbatim}
docker image ls
\end{verbatim}
\item
  To check what is running and which ports are occupied always run:

\begin{verbatim}
docker ps 
\end{verbatim}
\end{itemize}

\hypertarget{system-monitoring}{%
\section{System Monitoring}\label{system-monitoring}}

It is highly recommended that you monitor the systems processes during before and while running your processes. This can be done via the interactive system-monitor process-viewer and process-manager \textbf{htop}, which can be started by typing htop into the terminal and pressing enter.

Things to keep an eye on:

\begin{itemize}
\tightlist
\item
  How many threads are currently being used?
\item
  How much memory is available?
\end{itemize}

If your process is taking longer than expected, then this may be caused by either too many processes running in parallel, or insufficient memory being available. The later can result in swap memory being used, which will significantly slow down your processes.

\hypertarget{gpu-monitoring}{%
\section{GPU Monitoring}\label{gpu-monitoring}}

Another resource that requires monitoring are GPUs. A GPU can be monitored using the NVIDIA System Management Interface (nvididia-smi):

\begin{verbatim}
watch -n 0.1 nvidia-smi
\end{verbatim}

This interface will allow you to monitor the memory usage, volatile GPU utility, temperature and fan speed. If there is no memory available, then it is worth enquiring with the other individual using the GPU if they are using TensorFlow and have enabled memory growth. If memory growth has not been enabled, then TensorFlow will by default allocate the all available GPU memory to a task. See the following discussion for \href{https://stackoverflow.com/questions/34199233/how-to-prevent-tensorflow-from-allocating-the-totality-of-a-gpu-memory}{more information}.

When servers are situated within a non-air conditioned room it is also worth keeping an eye on the GPUs temperature, in particular when the server houses multiple GPUs.

  \bibliography{book.bib,packages.bib}

\end{document}
